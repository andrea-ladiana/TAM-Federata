%% ============================================================
%% Figure: Client-Aware Adaptive Weight
%% ============================================================

\begin{figure}[htbp]
\centering
\includegraphics[width=0.95\textwidth]{client_aware_w/output/panel_client_aware_w.pdf}
\caption{%
\textbf{Per-client adaptive weight dynamics in federated unsupervised learning with adversarial noise.}
%
\textbf{(A)}~Temporal evolution of the adaptive weight $w_c(t)$ (Eq.~\ref{eq:convex_comb}) for good clients (blue, $r=0.9$) and one attacker client receiving pure noise (vermillion dashed, $r=0$).
%
The weight update rule is based on normalized sign-agreement between the client's local Hebbian correlator $J^{(t)}_c$ and the server's reconstruction operator from the previous round (Section~\ref{subsec:plasticity_w}).
%
Good clients maintain stable non-zero weights ($w_{\mathrm{good}} \approx 0.4$--$0.6$), reflecting high-quality local data.
%
The attacker's weight rapidly converges to zero ($w_{\mathrm{att}} \to 0$ within $\sim$10 rounds), effectively down-weighting noisy contributions in the server aggregation.
%
Shaded regions represent standard error over $S=20$ independent seeds.
%
\textbf{(B)}~Server-side retrieval quality: per-archetype magnetization $m_k(t)$ (colored lines, $k=1,2,3$) and mean retrieval $\langle m \rangle$ (black line).
%
The system maintains high reconstruction fidelity ($\langle m \rangle \gtrsim 0.7$) despite the presence of one attacker, demonstrating robustness of the adaptive weighting scheme.
%
Parameters: $N=1000$ neurons, $K=3$ archetypes, $L=5$ clients (4 good + 1 attacker), $T=10$ rounds, $M=800$ examples/client/round, $\alpha_{\mathrm{EMA}}=0.5$.
}
\label{fig:client_aware_adaptive_weight}
\end{figure}


%% ============================================================
%% Main text (suggested placement after Section on plasticity/stability)
%% ============================================================

To validate the per-client adaptive weight mechanism described in Section~\ref{subsec:plasticity_w}, 
we conducted a federated unsupervised learning experiment in which one client (the \emph{attacker}) 
receives pure noise ($r=0$), while the remaining clients observe data drawn from the same latent 
archetypes at high quality ($r=0.9$).
%
Figure~\ref{fig:client_aware_adaptive_weight}A shows the temporal evolution of the adaptive 
weight $w_c(t)$ (Eq.~\ref{eq:convex_comb}) for both populations.
%
The weight update rule, based on normalized sign-agreement between the client's local 
Hebbian matrix and the server's reconstruction operator, successfully discriminates signal from noise: 
good clients stabilize at $w_{\mathrm{good}} \approx 0.4$--$0.6$, balancing local evidence with 
server guidance, while the attacker's weight collapses to near-zero within $\sim$10 rounds.

Importantly, this down-weighting of noisy clients preserves server-side reconstruction quality 
(Figure~\ref{fig:client_aware_adaptive_weight}B).
%
Despite the presence of one attacker among five clients (\SI{20}{\percent} contamination), 
the mean retrieval $\langle m \rangle$ remains above $0.7$ throughout training, indicating that 
the server's TAM dynamics (Section~\ref{sec:model}) successfully disentangle the $K=3$ latent 
archetypes from the aggregated—but adaptively weighted—Hebbian correlators.
%
Per-archetype magnetizations $m_k(t)$ (colored curves) exhibit stable convergence, with minor 
fluctuations attributable to stochastic sampling and the non-convex TAM energy landscape.

This experiment demonstrates that \emph{local} sign-agreement entropy, computed independently 
by each client without access to ground-truth labels, provides a reliable signal for data quality, 
enabling robust federated learning even under adversarial noise injection.
%
The approach generalizes naturally to heterogeneous client populations with varying $r_c$ 
(partial observations, label noise, distribution shift), offering a principled mechanism for 
\emph{plasticity-stability balance} in decentralized unsupervised settings.


%% ============================================================
%% Alternative compact version (if space is limited)
%% ============================================================

% To validate the adaptive weight mechanism (Section~\ref{subsec:plasticity_w}), we simulated 
% a federated scenario with $L=5$ clients, where one client receives pure noise ($r=0$) 
% and four observe high-quality data ($r=0.9$).
% %
% Figure~\ref{fig:client_aware_adaptive_weight}A shows that the sign-agreement-based weight 
% $w_c(t)$ successfully discriminates signal from noise: good clients stabilize at 
% $w \approx 0.4$--$0.6$, while the attacker's weight decays to zero within $\sim$10 rounds.
% %
% Despite \SI{20}{\percent} contamination, server reconstruction quality remains high 
% ($\langle m \rangle \gtrsim 0.7$, panel B), demonstrating robustness to adversarial noise.

